\chapter{Manuale installazione}
Si riassumono i requisiti necessari all'installazione del sistema e le modalità di configurazione.

\section{Ambiente}
Il sistema è stato sviluppato in ambiente \emph{Linux} e per la sua installazione ed esecuzione si consiglia il sistema operativo \emph{Ubuntu Linux ver. 14.04 LTS}. Il requisito non è vincolante, ma semplifica l'installazione e la gestione dei pacchetti necessari per le dipendenze. In termini di risorse il sistema richiede almeno 2 unità di calcolo a 64 bit e 2GB di memoria RAM disponibile.

\section{Dipendenze}
L'ambiente di esecuzione si compone di due elementi principali, il \textbf{database} contenente tutte le informazioni persistenti del sistema e la \textbf{logica applicativa} sviluppata con tecnologia \emph{Java}.

\subsection{Database}
E' necessario utilizzare un database \emph{PostgreSQL ver 9.3} o superiore. L'installazione può avvenire sia tramite il gestore dei pacchetti del sistema operativo che in modalità manuale. Oltre al pacchetto standard, sarà necessario installare due estensioni del database, in particolare:
\begin{itemize}
\item \textbf{PostGIS}: un'estensione che consente la memorizzazione di dati spaziali e fornisce altre funzioni di manipolazione di dati geografici (\url{http://postgis.net});
\item \textbf{PGRouting}: una libreria che estende ulteriormente le funzioni del database geospaziale introducendo algoritmi di \emph{routing} e procedure di supporto (\url{http://pgrouting.org}).
\end{itemize}
Nei sistemi \emph{Ubuntu Linux} l'installazione del servizio di database e di tutte le dipendenze può avvenire tramite l'esecuzione dei comandi:
\begin{verbatim}
sudo add-apt-repository ppa:georepublic/pgrouting-unstable
sudo apt-get update
sudo apt-get install postgresql-9.3-pgrouting
\end{verbatim}
Seguire le indicazioni presentate a riga di comando, in particolare per le prime operazioni in cui si aggiunge un \emph{repository} non ufficiale al gestore dei pacchetti.

\subsection{Java}
Per eseguire il codice con cui è stata sviluppata la logica applicativa del server, è necessario disporre di un ambiente \emph{Java Virtual Machine} compatibile. Si consiglia di installare il pacchetto \emph{Oracle Sun JDK ver. 1.7.0} secondo le modalità previste dal sistema operativo usato. Non si garantisce l'esecuzione del software in ambiente \emph{Java 8} nè con altre \emph{JVM} diverse da quella Oracle SUN.

Nello sviluppo è stato utilizzato il framework di supporto \emph{Play! Framework ver. 1.2.7}. Installare il pacchetto seguendo le indicazioni sul sito ufficiale (\url{https://www.playframework.com/documentation/1.2.7/install}). La modalità standard consiste nello scaricare l'archivio e scompattarlo in una posizione nota.
Le librerie necessarie all'esecuzione del codice sono incluse nel \emph{repository} del codice e saranno recuperate tramite i successivi passi di installazione.

\subsection{Repository}
Il codice sviluppato è disponibile tramite \emph{repository} pubblico \emph{GIT} all'indirizzo \url{https://github.com/tzorzan/path-server.git}. Clonare il \emph{branch master} in locale per ottenere una copia aggiornata del codice da eseguire.

\section{Configurazione}
Eseguire il \emph{restore} del database iniziale. Il file da cui eseguire il ripristino è presente nella posizione \emph{install/paths.backup} ed è in formato archivio. Eseguendo questa operazione si crea un database con la struttura corretta ed un set di dati di partenza. Il ripristino può essere eseguito tramite tool grafici o da linea di comando utilizzando le modalità previste da \emph{PostgreSQL} (\url{http://www.postgresql.org/docs/9.3/static/backup.html}).

A questo punto è necessario istruire l'applicazione fornendo il puntamento al database da utilizzare. Questo può essere fatto impostando la variabile d'ambiente \emph{DATABASE\_URL} prima di avviare l'applicazione. Impostare il valore sostituendo i parametri necessari:
\begin{verbatim}
export \
DATABASE_URL=postgres://<user>:<pwd>@<address>:<port>/<db_name>
\end{verbatim}
Impostare le credenziali di un utente che possiede diritti completi sullo schema, dato che alcune procedure richiedono la cancellazione e la creazione di alcune tabelle.

Per installare le dipendenze (librerie \emph{Java}) necessarie all'esecuzione è necessario posizionarsi nella directory base del progetto e lanciare il comando:
\begin{verbatim}
play deps
\end{verbatim}
Saranno scaricate tutte le dipendenze necessarie e salvate nella cartella \texttt{/lib} usata durante l'esecuzione del software.

\section{Avvio}
Avviare l'applicazione tramite il comando:
\begin{verbatim}
play run
\end{verbatim}
In questo modo si avvia il server e resta collegato al terminale in uso. 

Per avviare in \emph{background} il servizio e quindi lasciare il server attivo usare i comandi di avvio e spegnimento:
\begin{verbatim}
play start
play stop
\end{verbatim}

Il software utilizzerà come \emph{log} di sistema il file: \texttt{/logs/system.out}.

\section{Note}
E' necessario riconfigurare i client mobile affinchè puntino al nuovo server avviato. Questa operazione allo stato attuale può essere effettuata modificando il file sorgente \texttt{com.\-path3ar.\-model\-Utilities.java} e rieseguendo packaging ed installazione dell'applicazione.
