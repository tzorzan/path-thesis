La diffusione dei dispositivi \emph{smartphone} e l'accesso ad internet in mobilità sono ormai un dato di fatto nel panorama delle telecomunicazioni. La base di utenza si è allargata a dismisura, coprendo quasi tutte le fasce d'età così come l'infrastruttura tecnologica consente ormai di utilizzare questi mezzi in svariate situazioni permettendo di restare sempre ``connessi''. 

L'utilizzo della applicazioni mobili è ormai parte delle nostre attività quotidiane, sia ricreative che in ambito lavorativo. Le esigenze e le sfide per questo tipo di strumenti sono sempre più ambiziose: vogliamo applicazioni \emph{smart}, che ci rendano facili e accessibili le informazioni più complesse e che offrano una esperienza d'uso personalizzata e tagliata su misura sulle nostre richieste.

Il progetto \emph{PathS} si sviluppa in questo contesto, proponendo un'evoluzione dei sistemi di navigazione pedonale. L'idea di fondo è che la  proposta del tragitto più breve verso una destinazione non sia più l'unica informazione da fornire agli utenti, ma grazie ai recenti sviluppi tecnologici può essere integrata con un'insieme di elementi di contorno che possono rendere più coinvolgente e mirata l'esperienza di utilizzo.

Il progetto è stato coordinato nelle attività dal Prof. Claudio Palazzi e dalla Prof.ssa Ombretta Gaggi dell'Università desgli Studi di Padova. Nelle attività di analisi e approfondimento hanno collaborato i dottorandi Matteo Ciman e Armir Bujari, mentre le attività riguardanti l'applicazione \emph{client} sono state svolte dallo studente Stefano Tombolini.

In questo elaborato saranno presentati inizialmente il contesto e un insieme di progetti che per problematica affrontata o approccio utilizzato risultano affini e di interesse. Saranno introdotti alcuni riferimenti di letteratura in particolare riguardo la tematica dei \emph{Moving Object Databases}.

Il capitolo \ref{cap:client} fornirà un riassunto delle caratteristiche e delle funzioni offerte della componente \emph{client}, le quali risultano fondamentali per una visione complessiva del progetto.

I capitoli successivi documentano le attività di analisi e di implementazione del \emph{server PathS} nelle sue due macro-funzioni: ricezione e analisi dei campioni e calcolo dei percorsi pedonali.

L'ultimo capitolo fornirà un esempio dei risultati raggiunti e proporrà alcune aree di miglioramento così come sono state individuate durante lo svolgimento delle attività.
