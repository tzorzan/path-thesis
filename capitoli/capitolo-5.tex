La funzione di ricezione dei campioni ha richiesto la definizione di una nuova \emph{API} di comunicazione con le applicazioni \emph{client} ed in seguito sviluppare le componenti di manipolazione dei dati ricevuti per la relativa persistenza. Si presentano le specifiche tecniche e di implementazione di questi elementi.

\section{API}
Nel definire il dettaglio delle chiamate necessarie all'invio dei dati, si è valutato di prediligere i criteri di semplicità e adeguamento alle tecnologie attuali. Si è quindi optato per una \emph{API HTTP} e contenuto in formato JSON. La chiamata da utilizzare per l'invio dei dati risponde alla seguente specifica:

\begin{verbatim}
POST /api/data

BODY: 
{
    "type": "FeatureCollection",
    "features": [
        {
            "type": "Feature",
            "geometry": {
                "type": "Point",
                "coordinates": [
                    <longitude>,
                    <latitude>
                ]
            },
            "id": <unique_id>,
            "properties": {
                "timestamp": <timestamp>,
                "accuracy": <accuracy_value>,
                "labels": {
                    <label_name>: <label_value>,
                    ...
                }
            }
        },
        ...
    ],
    "properties": {
        "vote": <vote_number>
    }
}
\end{verbatim}
 Il formato definito consente di specificare una lista di elementi di campionamento e per ciascuno di esso le informazioni necessarie. Ogni campionamento è individuato da una chiave \emph{id} generata in modo randomico dal ciente seguendo lo standard UUID versione 4\footnote{\url{https://en.wikipedia.org/wiki/Universally_unique_identifier}}. Per ciascun campionamento sono indicati:
\begin{itemize}
	\item l'\emph{id} di riferimento;
	\item le coordinate espresse in \emph{latitudine} e \emph{longitudine};
	\item il dettaglio \emph{temporale} dell'avvenuto campionamento;
	\item l'\emph{accuratezza} del rilevamento GPS così come fornita dal dispositivo;
	\item le \emph{etichette} e i \emph{valori} per il campione rilevato dai sensori.
\end{itemize}
Tutti i valori sono numeri e stringhe in formato JSON. Per la struttura del contenuto si è preferito non adottare una forma libera ma piuttosto aderire ad uno standard già definito, ovvero GeoJSON\footnote{\url{http://geojson.org}}. In questo modo, pur consentendo una struttura semplice ed estendibile (il numero e il tipo di rilevazioni non è limitato), si è favorità l'inter-operabilità con altre librerie e servizi. Risulta inoltre più agevole una validazione del formato dei dati ricevuti.
Per le chiamata non è stato previsto alcun protocollo di autenticazione del \emph{client}, ritenendolo non necessario in questa fase di sviluppo del progetto.

\section{Implementazione}
Implemenazione semplificata dalle funzione del framework Play
Classe conroller;
Deserializzazione JSON

\begin{figure}[ht]
  \centering
  \includegraphics[width=\textwidth]{db-samples}
  \caption{\footnotesize{Diagramma ER modello dati campionamenti.}}
  \label{fig:paths-general}
\end{figure}
Persistenza JPA Model + schema ER

\section{Esempi}
Esempio invio campioni 
Un esempio di contenuto della chiamata è presente all'indirizzo \url{http://path-server.herokuapp.com/public/stub/path-samples.json}.

(screenshot map matching 1)
(screensho pah graphs)
Lista campioni