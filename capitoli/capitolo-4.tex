La parte cruciale del progetto \emph{PathS} si basa sulla gestione delle informazioni aggiuntive relative ai percorsi pedonali e la possibilità di sfruttarle per fornire un servizio alternativo agli utenti finali. La persistenza, la gestione e l'elaborazione di questi dati avviene nella componente \emph{server}. 
In questo capitolo saranno presentati i requisiti individuati per assolvere allo scopo del progetto e il modo in cui questi requisiti hanno guidato la definizione dell'architettura del software da sviluppare.

\section{Requisiti}
Come per l'altra componente, anche nel caso del server si è preso spunto dalla precedente versione del progetto (\emph{Path2.0}) e si è cercato di identificarne i limiti per poter essere adattato al nuovo contesto e agli scopi più ampi che sono stati definiti. Le situazioni e le necessità a cui si intende applicare il progetto sono del tipo:
\begin{itemize}
\item qual è il percorso meno rumoroso in quest’ora del giorno?
\item che percorso devo seguire per ottenere il maggior numero di tratti all’ombra?
\item quali sono i percorsi preferiti dagli utenti che hanno visitato questa zona?
\end{itemize}
Le risposte fornite dovranno sempre tenere conto della soddisfabilità della domanda (quindi fornire sempre un percorso valido) ma valutare anche le preferenze espresse dall’utente.

Dal punto di vista tecnologico la soluzione precedente è stata valutata non idonea come base di partenza, in particolare presentava le seguenti caratteristiche che ne rendevano difficile una possibile evoluzione:
\begin{itemize}
\item il sistema era stato basato sul concetto dei percorsi accessibili e la ricerca del \emph{routing} esclusivamente come riutilizzo dei tratti noti con questa caratteristica;
\item vi è un forte accoppiamento e dipendenza dalle API Google Maps, i cui termini di utilizzo sono cambiati nel tempo e il modo in cui sono integrate è una forzatura inefficente (interrogazioni multiple per soddisfare una singola richiesta di routing);
\item il formato di comunicazione con il client era specifico e non adattabile al nuovo contesto senza una profonda rielaborazione;
\item il modello di persistenza delle informazioni non era adatto a gestire il nuovo \emph{set} di informazioni aggiuntive.
\end{itemize}

\begin{figure}[ht]
  \centering
  \includegraphics[width=\textwidth]{paths-general}
  \caption{\footnotesize{Schema generale del progetto PathS.}}
  \label{fig:paths-general}
\end{figure}

Considerati questi aspetti, ed in particolare le difficoltà nell'estendere il progetto esistente;  con il gruppo di lavoro si è cercato di ridefinire i requisiti del sistema ponendo l'attenzione su alcuni punti principali, ovvero:
\begin{itemize}
\item \textbf{raccolta campioni}: il sistema deve ricevere i campioni dalle applicazioni \emph{client} in un formato ben definito e dalle possibilità di estensione;
\item \textbf{persistenza}: il sistema deve memorizzare in modo efficiente e coerente le informazioni ricevute relative a campioni di diversa tipologia;
\item \textbf{associazione dei campioni ai percorsi}: il sistema deve eseguire l'operazione di associazione delle informazioni raccolte dai sistemi client (sensori) alle informazioni di geolocalizzazione e cartografia (tratti stradali);
\item \textbf{routing}: il sistema deve implementare un servizio di calcolo dei percorsi pedonali complesso che rielabori le informazioni precedentemente raccolte;
\item \textbf{interfacciamento con il client}: il sistema deve comunicare con le applicazioni \emph{mobile}  e i client \emph{desktop} fornendo le informazioni necessarie alla navigazione.
\end{itemize}

Oltre a questi requisiti fondamentali per il funzionamento del sistema, sono stati individuati altri criteri preferibili di carattere qualitativo da considerare nella progettazione e implementazione del software. Si possono riassumere in:
\begin{itemize}
\item \textbf{suddivisione delle responsabilità}: definire in modo specifico le funzioni svolte da ciascun componente del sistema, evitando elementi che svolgono funzioni troppo complesse o che accoppiano troppi elementi. Questo approccio consente una migliore testabilità delle sotto-componenti e la possibilità di rivedere e riprogettare alcuni dettagli senza dover manipolare l'intero prodotto software;
\item \textbf{astrazione}: identificare la funzione logica di ciascun componente concentrandosi sulla sua interfaccia prima di proseguire nell'implementazione. Questo consente di delineare con precisione il ruolo che dovrà svolgere, precondizioni,risultati attesi e i confini concettuali in cui opera. Rispettare questo criterio rende più facile lo sviluppo e il miglioramento dell'implementazione dei componenti senza doverne ridefinire la funzione logica e riducendo le ripercussioni sugli altri elementi dell'architettura;
\item \textbf{estensibilità}: rendere agevole sia in termini architetturali che implementativi la possibilità di migliorare ed estendere il sistema. L'obiettivo del progetto è stato volutamente limitato ad un primo risultato tangibile, considerando però che vi sia la possibilità di migliorare ed estendere le sue parti in modo facile e coerente con la base che si andrà a sviluppare.
\end{itemize}

\section{Componenti}
Basandosi sulle funzioni che deve svolgere il server, sono state identificate le componenti logiche principali da sviluppare. Per ciascun caso d'uso, si presenta in dettaglio il ruolo delle componenti e il modo in cui assolvono al raggiungimento del risultato.

\begin{figure}[ht]
  \centering
  \includegraphics[width=\textwidth]{paths-componenti}
  \caption{\footnotesize{Macro componenti del del Server PathS.}}
  \label{fig:paths-componenti}
\end{figure}

\subsection{Ricezione dei campioni}
La prima componente che partecipa alla funzione di ricezione dei campioni deve dialogare con il client \emph{mobile}. Si è pensato di definire una \textbf{API} su protocollo \emph{HTTP} da utilizzare per tutte le chiamate di invio dati; in questo modo utilizzando un formato semplice che rispetta gli attuali \emph{standard de-facto}, risulta facilmente implementabile con qualsiasi tecnologia \emph{client} ed eventualmente da altre sorgenti dati future. Il componente \textbf{API} si occupa quindi di deserializzare i dati della richiesta e riorganizzarla per la persistenza. Il salvataggio avviene su apposito \textbf{Database} ma mediato da un componente intermedio di \textbf{Model}. Risultano quindi più agevoli le operazioni di interrogazione e persistenza, nonchè la leggibilità e semplicità dell'implementazione.

\subsection{Elaborazione dei campioni}
I campioni così come sono ricevuti dai client sono in formato grezzo e non consentono di sfruttare le inormazioni in esse contenute. Il passo principale per l'utilizzo dei dati è quello di associare ciascun dato ad un segmento della rete di trasporto. Tutto questo procedimento è sintetizzato con l'espressione \textbf{Map Matching} che è di fatto l'algoritmo che esegue tale associazione. Il \emph{matching} viene eseguito sia in modo autonomo dal server che tramite invocazioni dell'\textbf{API}. Il risultato delle esecuzioni dell'algoritmo viene comunque salvato a \textbf{database} tramite le apposite classi di \textbf{model}.

\subsection{Calcolo Percorsi}
Lo scopo pensato per l'utilizzo delle informazioni raccolte è quello di fornire servizi \emph{routing} alternativi, che tengano conto dei dati raccolti per suggerire percorsi che vanno oltre la semplice regola del percorso più breve. Per questo caso d'uso son quindi coinvolti i componenti:
\begin{itemize}
\item \textbf{API} per definire un formato con cui i client (\emph{app} o \emph{browser}) richiedono un percorso;
\item implementazione di un algoritmo di \textbf{routing} che esegue il calcolo effettivo;
\item l'algoritmo utilizzi le classi di \textbf{model} per accedere alle informazioni aggiuntive utilizzate nel calcolo.
\end{itemize}

\subsection{Accesso da Browser}
Per tutte le funzionalità principali del server si è pensato di dare un accesso di \emph{monitoraggio} tramite \emph{web browser}. In questo modo è possibile accedere facilmente alle informazioni gestite dal sistema, tramite visualizzazioni grafiche su mappa. La presentazione dei dati recuperati tramite le classi \textbf{model} è implementata con pagine \emph{html} e linguaggio \emph{Javascript} eseguito lato \emph{client}. Queste componenti possono essere raggruppate con la definizione di \textbf{view}.

\section{Tecnologie}
Per la realizzazione del progetto \emph{server} sono state selezionate alcune tecnologie e librerie utilizzate in modo trasversale per lo sviluppo dell'applicazione. I criteri adottati in questa scelta si riassumono in:
\begin{itemize}
\item preferenza per gli strumenti \emph{Open Source} facilitando accesso al software, la gestione delle licenze e lo sviluppo di eventuali modifiche;
\item preferenza per gli strumenti di larga adozione, per i quali sono disponibili \emph{on-line} documentazione ed esperienza diretta degli utenti,
\item preferenza per gli strumenti già adottati in altri progetti, in modo da ridurre i tempi di apprendimento e di configurazione.
\end{itemize}

Il risultato di questa selezione sono stati:
\begin{itemize}
\item \textbf{Framework Play!} (\url{https://www.playframework.com}): utilizzato per la struttura principale dell'applicazione Java. L'adozione di questo framework (già utilizzato anche in esperienze lavorative) ha permesso un rapido \emph{setup} dell'architettura \emph{Model-View-Controller} di base del server. Inoltre risultano semplificate alcune funzioni tra cui il \emph{mapping} delle richieste \emph{HTTP} previste dall'\emph{API}, la gestione della persistenza e l'\emph{Object Relational Mapping} tramite la specifica \emph{JPA} e l'implementazione \emph{Hibernate}. Il framework fornisce inoltre un facile motore di \emph{templating} per lo sviluppo delle pagine di presentazione.
\item \textbf{PostgresSQL} (\url{http://www.postgresql.org}): è il \emph{RDBMS} selezionato in quanto tra i prodotti open source più validi e diffusi allo stato attuale. Risulta inoltre particolarmente adatto con le funzioni \emph{GIS} aggiuntive fornite dalle librerie presentate in seguito e fondamentali per l'implementazione del progetto.
\item \textbf{LeafletJS} (\url{http://leafletjs.com}): libreria Javascript utilizzata per la presentazione delle mappe interattive. E' un prodotto open source altamente configurabile e di facile integrazione. E' risultato lo strumento ideale per presentare i dati in questa forma visuale senza legarsi ad altri servizi esterni. Supporta la visualizzazione da \emph{mobile} ed è particolarmente utile per la rappresentazione di \emph{layer} sovrapposti contenenti informazioni diverse.
\item \textbf{Bootstrap} (\url{http://getbootstrap.com}): Uno dei framework \emph{HTML/CSS/JavaScript} più diffusi in assoluto, utilizzato per semplificare la realizzazione delle pagine web. Ha permesso la realizzazione di pagine web con \emph{design responsive} e graficamente gradevoli.
\end{itemize}
