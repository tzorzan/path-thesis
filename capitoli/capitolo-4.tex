Introduzione alla componente \emph{server} di \emph{PathS}. 

\section{Requisiti}
Dettaglio dei requisiti per la parte server: raccolta campioni, persistenza, interfacciamento con il client, associazione dei campioni ai percorsi, servizio di routing.

\section{Componenti}
Identificazione delle componenti logiche del server sulla base delle funzioni che deve svolgere.
\subsection{Ricezione dei campioni}
La componente di ricezione campioni deve colloquiare con il client. Necessità di definire un formato chiaro e \emph{standard}, facilmente implementabile con qualsiasi tecnologia \emph{client} ed eventualmente da altri sorgenti dati.
\subsection{Elaborazione dei campioni}
I campioni così come sono ricevuti dai client sono in formato grezzo e non consentono di sfruttare le inormazioni in esse contenute. Il passo principale per l'utilizzo dei dati è quello di associare ciascun dato ad un segmento della rete di trasporto. 
\subsection{Interrogazione}
Lo scopo pensato per l'utilizzo delle informazioni raccolte è quello di fornire servizi \emph{routing} alternativi, che tengano conto dei dati raccolti per suggerire percorsi che vanno oltre la semplice regola del percorso più breve. Il componente deve quindi:
\begin{itemize}
\item definire un formato di colloquio con il client per i percorsi;
\item implementare un algoritmo di routing;
\item modificare l'algoritmo di routing affinché consideri le informazioni aggiuntive.
\end{itemize}

\section{Tecnologie e linee guida}
Tecnologie e librerie selezionate per l'applicazione globale al sistema server (Play! Framework, PostgresSQL, LeafletJS, Bootstrap). Presentazione dei criteri non funzionali con i quali è stato sviluppato il server (suddivisione delle responsabilità, astrazione, estensibilità).