\documentclass[a4paper, 12pt, twoside, openright]{book}
\usepackage[english]{babel}
\usepackage[T1]{fontenc}
\usepackage[latin1]{inputenc}
\usepackage{fancyhdr}
\usepackage{float}
\usepackage{graphicx}
\usepackage{wrapfig}
%------------------------------ colors
\usepackage[usenames,dvipsnames,table]{xcolor} % use colors on table and more
\definecolor{333}{RGB}{51, 51, 51} % define custom color
%------------------------------ source code
\usepackage{listings}
\lstset{
  basicstyle=\footnotesize\sffamily,
  commentstyle=\itshape\color{gray},
  captionpos=b,
  frame=shadowbox,
  language=HTML,
  rulesepcolor=\color{333},
  tabsize=2
}
%------------------------------ define Abstract environment, missing in the 'book' class
\newenvironment{abstract}{\cleardoublepage \null \vfill \begin{center}\bfseries\abstractname \end{center}}{\vfill\null}
\addto\captionsenglish{\renewcommand*\abstractname{Sommario}} % change Abstract title
%------------------------------ active url
\usepackage{url}
\renewcommand{\UrlFont}{\color{black}\small\ttfamily}
\usepackage[colorlinks=true, linkcolor=black, citecolor=black, urlcolor=black]{hyperref} % active ref
%------------------------------ macros
\newcommand{\sectionname}{Section} % define Section ref
\newcommand{\subsectionname}{Sub-section} % define Sub-section ref
\renewcommand*\arraystretch{1.4} % tables padding


\begin{document}
\frontmatter

\begin{titlepage} %------------------------------ TITLE PAGE
\begin{center}
\vbox to0pt{\vbox to\textheight{\vfill \includegraphics[width=11.5cm]{images/unipd-light.png} \vfill}\vss}

\hspace{0.5cm}
\begin{minipage}{.20\textwidth}
  \includegraphics[height=2.5cm]{images/unipd-bn.png}
\end{minipage}\begin{minipage}{.90\textwidth}
  \begin{table}[H]
  \begin{tabular}{l}
  \scshape{\Large{\bfseries{Universit� degli Studi di Padova}}} \\
  \hline \\
  \scshape{\Large{Dipartimento di Matematica}} \\
  \end{tabular}
  \end{table}
\end{minipage}

\vspace{1cm}
\emph{\Large{Laurea~Magistrale~in~Informatica}} \\
\vspace{0.5cm}
\emph{\Large{Indirizzo~Sistemi}} \\
\vspace{1.5cm}
\scshape{\Large{\bfseries{PathS Server: ...}}} \\
\vspace{0.2cm} \linespread{1} \scshape{\large{\bfseries{(sottotitolo tesi)}}}
\end{center}

\vfill
\begin{normalsize}
\begin{flushleft}
  \hspace{45pt} \textit{Laureando} \hspace{160pt} \textit{Relatore}\\
  \vspace{5pt}
  \hspace{30pt} \large{\textbf{Tobia Zorzan}} \hspace{70pt} \large{\textbf{Prof. Claudio Palazzi}}\\
  \vspace{10pt}
  \hspace{260pt} \normalsize{\textit{Co-relatore}}\\
  \vspace{5pt}
  \hspace{200pt} \large{\textbf{Prof.ssa Ombretta Gaggi}}
\end{flushleft}
\end{normalsize}

\vfill
\begin{center}
\hspace{-0.2cm}
\line(1, 0){360}

\textsc{Anno Accademico 2014/2015}
\end{center}
\end{titlepage}


\cleardoublepage % make left page blank
\thispagestyle{empty} %------------------------------ DEDICA

\null
\vspace{2cm}
\begin{flushright}
A ...
\end{flushright}
\vfill

\begin{quote}
  Quote

  \textit{Author}
\end{quote}
\vfill
\null


\begingroup %------------------------------ CONTENTS
  \makeatletter
  \let\ps@plain\ps@empty
  \makeatother
  \tableofcontents
  \clearpage
\endgroup


\begin{abstract} %------------------------------ ABSTRACT
\markboth{}{} % remove header
\thispagestyle{empty}
I recenti sviluppi dei dispositivi \emph{smartphone}, in particolare l'inclusione di sensori sempre pi� evoluti, ha portato alla definizione di un nuovo paradigma denominato \emph{Web\textsuperscript{2}} (\emph{Web Squared}). La diffusione capillare di questi strumenti consente di creare una rete di sensori distribuiti in grado di raccogliere una ingente mole di dati. I dati stessi possono essere quindi elaborati e fornire il supporto a nuovi servizi che non si basano sulle sole informazioni \emph{statiche} ma su un contesto \emph{dinamico} modificato dall'interazione degli utenti. Il progetto \emph{PathS} vuole essere un prototipo che segue questo paradigma. La sua componente \emph{client} raccoglie informazioni aggiuntive sui percorsi pedonali percorsi dagli utenti. La parte \emph{server} elabora le informazioni raccolte e mantiene una base di dati in grado di supportare servizi di interrogazione che adottano criteri aggiuntivi a quelli tradizionali.
Si espone quindi quali sono stati i criteri e le valutazioni eseguite nella realizzazione del componente \emph{PathS Server}, i risultati raggiunti e le possibili aree di miglioramento individuate.
\end{abstract}


\mainmatter

\iffalse
- Introduzione: riassunto dei capitoli e dei contenuti
- Contesto e Lavori Correlati
  - Web Squared
  - Related Works
- PathS Client
  - Obiettivi
  - Tecnologie
  - Stato dell'arte
- PathS Server
  - requisiti
  - componenti
  - Tecnologie e linee guida
    - Play! Framework
    - PostgreSQL
    - Libreria LeafletJS
- Ricezione dei campioni
  - formato GeoJson
  - api Stub
- Elaborazione dei campioni
  - servizi di mappe
    - Google Maps
    - OpenStreet Map
      - Overpass API
  - persistenza ed eleaborazione dei dati GIS
    - PostGIS
    - JTS
    - PGRouting
 - Algoritmo di Map Matching
    - ST Map Matching
    - Implementaizone Algoritmo
      - step 1
      - step 2
      - step 3
- Servizi di routing
  - Obiettivi
  - Servizio Map Quest
  - Shortest Path (Dijkstra)
  - Percorsi particolari
- Risultati
- Miglioramenti ed Evoluzioni

\fi

\chapter{Introduzione} %------------------------------ INTRODUZIONE
\thispagestyle{empty}

Introduzione al progetto. Presentazione dello scopo principale e degli attori coinvolti. Rassegna del contenuto dei capitoli.

\chapter{Contesto e lavori correlati} %------------------------------ CONTESTO
\thispagestyle{empty}

\section{Web\textsuperscript{2}}
Presentazione del paradigma Web\textsuperscript{2} \cite{web2palazzi} ed esempi di applicazione. Opportunistic sensing e applicazione al progetto.

\section{Lavori correlati}
Presentazione progetto Path 1.0 e valutazioni. Motivazioni sviluppo nuova architettura. 
Accenno altri progetti simili o correlati.

\chapter{PathS Client} %------------------------------ PATHS CLIENT
\thispagestyle{empty}

Introduzione alla componente \emph{client} di \emph{PathS}. 

\section{Requisiti}
Presentazione degli obiettivi riguardanti la parte client. Raccolta dei campioni tramite sensori, offrire servizio di routing, coinvolgimento utente (tramite AR).

\section{Tecnologie}
Riassunto delle tecnologie adottate per la realizzazione della componente \emph{client}: AndroidSDK, Wikitude.

\section{Stato dell'arte}
Risultati ottenuti con la parte client. Funzionamento ed eventuali limiti riscontrati.

\chapter{PathS Server} %------------------------------ PATHS SERVER
\thispagestyle{empty}

Introduzione alla componente \emph{server} di \emph{PathS}. 

\section{Requisiti}
Dettaglio dei requisiti per la parte server.

\section{Componenti}
Identificazione delle componenti logiche del server sulla base delle funzioni che deve svolgere.
\subsection{Ricezione dei campioni}
\subsection{Elaborazione dei campioni}
\subsection{Interrogazione}

\section{Tecnologie e linee guida}
Tecnologie e librerie selezionate per l'applicazione globale al sistema server (Play! Framework, PostgresSQL, LeafletJS, Bootstrap). Presentazione dei criteri non funzionali con i quali � stato sviluppato il server (suddivisione delle responsabilit�, astrazione, estensibilit�).

\backmatter

\begingroup %------------------------------ BIBLIOGRAPHY
  \makeatletter
  \let\ps@plain\ps@empty
  \makeatother
  \bibliography{tesi}
  \addcontentsline{toc}{chapter}{Bibliography}
  \bibliographystyle{ieeetr} % sort in order of appearance
\endgroup
\end{document} 