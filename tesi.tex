\documentclass[a4paper, 12pt, twoside, openright]{book}
\usepackage[utf8]{inputenc}
\usepackage[italian]{babel}
\usepackage[T1]{fontenc}
\usepackage{import}
\usepackage{fancyhdr}
\usepackage{float}
\usepackage{graphicx}
\graphicspath{ {images/} }
\usepackage{caption}
\usepackage{subcaption}
\usepackage{wrapfig}
\usepackage{diagbox}
\usepackage[style=numeric, backend=biber]{biblatex}
\addbibresource{tesi.bib}
\usepackage{amsfonts}
\usepackage[toc,page]{appendix}
\renewcommand{\appendixtocname}{Appendice}
\renewcommand{\appendixpagename}{Appendice}

%------------------------------ colors
\usepackage[usenames,dvipsnames,table]{xcolor} % use colors on table and more
\definecolor{333}{RGB}{51, 51, 51} % define custom color
%------------------------------ source code
\usepackage{listings}
\lstset{
  basicstyle=\footnotesize\sffamily,
  commentstyle=\itshape\color{gray},
  captionpos=b,
  frame=shadowbox,
  language=HTML,
  rulesepcolor=\color{333},
  tabsize=2
}
%------------------------------ define Abstract environment, missing in the 'book' class
\newenvironment{abstract}{\cleardoublepage \null \vfill \begin{center}\bfseries\abstractname \end{center}}{\vfill\null}
\addto\captionsitalian{\renewcommand*\abstractname{Sommario}} % change Abstract title
%------------------------------ active url
\usepackage{url}
\renewcommand{\UrlFont}{\color{black}\small\ttfamily}
\usepackage[colorlinks=true, linkcolor=black, citecolor=black, urlcolor=black]{hyperref} % active ref
%------------------------------ macros
\newcommand{\sectionname}{Sezione} % define Section ref
\newcommand{\subsectionname}{Sottosezione} % define Sub-section ref
\renewcommand*\arraystretch{1.4} % tables padding


\begin{document}
\frontmatter

\begin{titlepage} %----------------------------------- INTESTAZIONE
\begin{center}
\vbox to0pt{\vbox to\textheight{\vfill \includegraphics[width=11.5cm]{unipd-light.png} \vfill}\vss}

\hspace{0.5cm}
\begin{minipage}{.20\textwidth}
  \includegraphics[height=2.5cm]{unipd-bn.png}
\end{minipage}\begin{minipage}{.90\textwidth}
  \begin{table}[H]
  \begin{tabular}{l}
  \scshape{\Large{\bfseries{Università degli Studi di Padova}}} \\
  \hline \\
  \scshape{\Large{Dipartimento di Matematica}} \\
  \end{tabular}
  \end{table}
\end{minipage}

\vspace{1cm}
\emph{\Large{Corso~di~Laurea~Magistrale~in~Informatica}} \\
\vspace{0.5cm}
\emph{\Large{Indirizzo~Sistemi}} \\
\vspace{1.5cm}
\scshape{\Large{\bfseries{PathS: un servizio Location-Based per percorsi alternativi}}} \\
\vspace{0.2cm} \linespread{1} \scshape{\large{\bfseries{(PathS: a Location-Based service for alternative routes)}}}
\end{center}

\vfill
\begin{normalsize}
\begin{flushleft}
  \hspace{45pt} \textit{Laureando} \hspace{160pt} \textit{Relatore}\\
  \vspace{5pt}
  \hspace{30pt} \large{\textbf{Tobia Zorzan}} \hspace{70pt} \large{\textbf{Prof. Claudio Palazzi}}\\
  \vspace{10pt}
  \hspace{260pt} \normalsize{\textit{Co-relatore}}\\
  \vspace{5pt}
  \hspace{200pt} \large{\textbf{Prof.ssa Ombretta Gaggi}}
\end{flushleft}
\end{normalsize}

\vfill
\begin{center}
\hspace{-0.2cm}
\begin{center}
\line(1,0){360}
\end{center}

\textsc{Anno Accademico 2014/2015}
\end{center}
\end{titlepage}


%-\cleardoublepage % make left page blank
%-\thispagestyle{empty} %----------------------------- DEDICA

%-\null
%-\vspace{2cm}
%-\begin{flushright}
%-A ...
%-\end{flushright}
%-\vfill

%-\begin{quote}
%-  Quote

%-  \textit{Author}
%-\end{quote}
%-\vfill
%-\null


\begingroup %----------------------------------------- CONTENTS
  \makeatletter
  \let\ps@plain\ps@empty
  \makeatother
  \tableofcontents
  \clearpage
\endgroup


\begin{abstract} %------------------------------------ ABSTRACT
\markboth{}{} % remove header
\thispagestyle{empty}
I recenti sviluppi dei dispositivi \emph{smartphone}, in particolare l'inclusione di sensori sempre più evoluti, ha portato alla definizione di un nuovo paradigma denominato \emph{Web\textsuperscript{2}} (\emph{Web Squared}). La diffusione capillare di questi strumenti consente di creare una rete di sensori distribuiti in grado di raccogliere una ingente mole di dati. I dati stessi possono essere quindi elaborati e fornire il supporto a nuovi servizi che non si basano sulle sole informazioni \emph{statiche} ma su un contesto \emph{dinamico} modificato dall'interazione degli utenti. Il progetto \emph{PathS} vuole essere un prototipo che segue questo paradigma. La sua componente \emph{client} raccoglie informazioni aggiuntive sui percorsi pedonali percorsi dagli utenti. La parte \emph{server} elabora le informazioni raccolte e mantiene una base di dati in grado di supportare servizi di interrogazione che adottano criteri aggiuntivi a quelli tradizionali.
Si espone quindi quali sono stati i principi e le valutazioni eseguite nella realizzazione del componente \emph{PathS Server}, i risultati raggiunti e le possibili aree di miglioramento individuate.
\end{abstract}


\mainmatter

\chapter{Introduzione} %------------------------------ INTRODUZIONE
\thispagestyle{empty}
\import{capitoli/}{capitolo-1.tex}

\chapter{Contesto e lavori correlati} %--------------- CONTESTO
\thispagestyle{empty}
\import{capitoli/}{capitolo-2.tex}

\chapter{PathS Client} %------------------------------ PATHS CLIENT
\thispagestyle{empty}
\import{capitoli/}{capitolo-3.tex}

\chapter{PathS Server} %------------------------------ PATHS SERVER
\thispagestyle{empty}
\import{capitoli/}{capitolo-4.tex}

\chapter{Ricezione dei campioni} %-------------------- RICEZIONE CAMPIONI
\thispagestyle{empty}
\import{capitoli/}{capitolo-5.tex}

\chapter{Elaborazione dei campioni} %----------------- ELABORAZIONE CAMPIONI
\thispagestyle{empty}
\import{capitoli/}{capitolo-6.tex}

\chapter{Routing} %----------------------------------- ROUTING
\thispagestyle{empty}
\import{capitoli/}{capitolo-7.tex}

\chapter{Conclusione} %------------------------------- CONCLUSIONE
\thispagestyle{empty}
\import{capitoli/}{capitolo-8.tex}

\begin{appendices}
\chapter{Manuale installazione} %--------------------- MANUALE INSTALLAZIONE
Requisiti e modalità di installazione.
\end{appendices}

\backmatter

\begingroup %----------------------------------------- BIBLIOGRAFIA
  \makeatletter
  \let\ps@plain\ps@empty
  \makeatother
  \printbibliography[
    heading=bibintoc,
    title={Bibliografia} ]
\endgroup

\end{document} 